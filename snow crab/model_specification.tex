% Options for packages loaded elsewhere
\PassOptionsToPackage{unicode}{hyperref}
\PassOptionsToPackage{hyphens}{url}
%
\documentclass[
]{article}
\usepackage{lmodern}
\usepackage{amssymb,amsmath}
\usepackage{ifxetex,ifluatex}
\ifnum 0\ifxetex 1\fi\ifluatex 1\fi=0 % if pdftex
  \usepackage[T1]{fontenc}
  \usepackage[utf8]{inputenc}
  \usepackage{textcomp} % provide euro and other symbols
\else % if luatex or xetex
  \usepackage{unicode-math}
  \defaultfontfeatures{Scale=MatchLowercase}
  \defaultfontfeatures[\rmfamily]{Ligatures=TeX,Scale=1}
\fi
% Use upquote if available, for straight quotes in verbatim environments
\IfFileExists{upquote.sty}{\usepackage{upquote}}{}
\IfFileExists{microtype.sty}{% use microtype if available
  \usepackage[]{microtype}
  \UseMicrotypeSet[protrusion]{basicmath} % disable protrusion for tt fonts
}{}
\makeatletter
\@ifundefined{KOMAClassName}{% if non-KOMA class
  \IfFileExists{parskip.sty}{%
    \usepackage{parskip}
  }{% else
    \setlength{\parindent}{0pt}
    \setlength{\parskip}{6pt plus 2pt minus 1pt}}
}{% if KOMA class
  \KOMAoptions{parskip=half}}
\makeatother
\usepackage{xcolor}
\IfFileExists{xurl.sty}{\usepackage{xurl}}{} % add URL line breaks if available
\IfFileExists{bookmark.sty}{\usepackage{bookmark}}{\usepackage{hyperref}}
\hypersetup{
  pdftitle={Size-based Population Model to Southern Gulf of Saint Lawrence Snow Crab},
  pdfauthor={Tobie Surette},
  hidelinks,
  pdfcreator={LaTeX via pandoc}}
\urlstyle{same} % disable monospaced font for URLs
\usepackage[margin=1in]{geometry}
\usepackage{longtable,booktabs}
% Correct order of tables after \paragraph or \subparagraph
\usepackage{etoolbox}
\makeatletter
\patchcmd\longtable{\par}{\if@noskipsec\mbox{}\fi\par}{}{}
\makeatother
% Allow footnotes in longtable head/foot
\IfFileExists{footnotehyper.sty}{\usepackage{footnotehyper}}{\usepackage{footnote}}
\makesavenoteenv{longtable}
\usepackage{graphicx,grffile}
\makeatletter
\def\maxwidth{\ifdim\Gin@nat@width>\linewidth\linewidth\else\Gin@nat@width\fi}
\def\maxheight{\ifdim\Gin@nat@height>\textheight\textheight\else\Gin@nat@height\fi}
\makeatother
% Scale images if necessary, so that they will not overflow the page
% margins by default, and it is still possible to overwrite the defaults
% using explicit options in \includegraphics[width, height, ...]{}
\setkeys{Gin}{width=\maxwidth,height=\maxheight,keepaspectratio}
% Set default figure placement to htbp
\makeatletter
\def\fps@figure{htbp}
\makeatother
\setlength{\emergencystretch}{3em} % prevent overfull lines
\providecommand{\tightlist}{%
  \setlength{\itemsep}{0pt}\setlength{\parskip}{0pt}}
\setcounter{secnumdepth}{-\maxdimen} % remove section numbering

\title{Size-based Population Model to Southern Gulf of Saint Lawrence Snow Crab}
\author{Tobie Surette}
\date{}

\begin{document}
\maketitle

\hypertarget{population-models}{%
\subsection{Population models}\label{population-models}}

Population models are mathematical constructs which attempt to
quantitatively describe how abundance indices are observed to vary over
time.

When successful, models capture some essence of the variation observed
in the data, providing some insight on underlying biological processes,
the impact of fishing activities, which can be used to predict future
dynamics of the population.

The complexity of these models can range from the very simple to the
complex and very detailed.

In this report, we present a size-based population model of the annual
dynamics of the Southern Gulf of Saint Lawrence snow crab population.

Though size-based, crab are partitioned by instar, i.e.~the number of
benthic moults crab have undergone, allowing for the specification of
biological effects based on instar, as well as on size or year.

Instars are identified using a structured growth process which models
the expected growth-at-moult between instars and its variation, which is
used to fit a set of modes, the product of instar growth constraints,
observed in the annual size-frequency data.

Biological processes included in the model are growth-at-moult,
population recruitment, natural and fishing-related mortality,
skip-moulting probabilities and moult-to-maturity probabilities.

Sampling-related processes are incorporated in the form of survey trawl
size-selectivity curves as well as global year effects, intended to shed
light on potential differences in catchability between survey years,
which have been associated with vessel changes, as well as other survey
sampling changes.

For the purposes of this study, the snow crab population is assumed to
be closed, which is a reasonnble assumption as the stock is largely
bounded by the Gaspé Pensinsula to the North, Cape Breton Island to the
Southeast, and the deep warm waters of the Laurentian Channel between
the two, though limited movement of crab along the margins of Gaspé and
Cape Breton are known to occur.

Limits of population models: Population models such as those presented
here are often over-parameterized, meaning that in some way the amount
of information being estimated exceeds the inherent information content
of the data. As such, such analyses often yield multiple solutions,
representing potentially very different biological states, which explain
the data to a similar degree.

Often, external biological information or assumptions are imposed on the
model to constrain the parameter space to solutions which are more
consistent with biological knowledge.

Population model development as such is generally a protracted process
of many iterations of adding or removing features, testing the model
against a suite of diagnostics, such as residual analyses, predictive
tests, cross-validations.

From these, the behaviour of the model under various assumptions becomes
known, and experience and a feelings regarding its robustness is gained.

In this context, the model described here is not presented as a fait
accompli, but rather as a presentation of the strategy and as a
prototype which has the right balance of simplicity and complexity to
provide, in the short-term, improvements in:

\begin{itemize}
\tightlist
\item
  the retroactive estimation and prediction of annual variations in
  skip-moulting and moult-to-maturity probabilities.
\item
  the prediction of future population dynamics, specifically fishery
  recruitment.
\item
  relative catchability estimates between different survey years, or
  survey vessels, which may provide a means of retroactively
  standardizing the snow crab survey abundance and biomass time series.
\end{itemize}

As side-benefits, the proposed modeling framework may also be expected
to yield:

\begin{itemize}
\tightlist
\item
  estimates of annual growth estimates, in the form of mean instar size
  estimates.
\item
  fishery-independent estimates of fishing mortality, including discard
  mortality.
\item
  a framework for modeling spatially-reference stock dynamics.
\end{itemize}

In this way, the population model represent a synthesis of various
biological and sampling processes.

Thus, population models can potentially provide information regarding a
slew of biological and survey effects and their interaction, although
this interaction often means that various solutions may come about which
explain the data to a similar degree \ldots{} i.e.~there may be
competing solutions, different suppositions, which may explain the
observed patterns in the data to a similar degree. So the model requires
good run of diagnostics and tests of robustness in order to explore the
limits and stability of any identified solutions.

\hypertarget{to-do}{%
\subsection{To do:}\label{to-do}}

\begin{itemize}
\item
  Depth-based estimation.
\item
  Implement fishing mortality effect : a scaled-logistic-type effect
  which kicks in around 95mm.
\item
  a\_year / (1 + exp(4\emph{m}(x - 95))), scale varies by year.
\item
  skip-moulting variation by year : global year increase (i.e.~not
  instar specific).
\item
  Remove superfluous p\_mat and p\_skip row/column
\item
  Add interaction error / mortality variability
\item
  Add vessel effect
\item
  Add mature growth modification\\
\item
  Display histograms with immatures y and mature y + 1
\item
  Add annual global scaling effect.
\item
  Improve instar variance fit for matures.
\item
  Improve growth mean fit for immature instars VIII and IX.
\item
  Implement log\_mu\_year which better controls variability within
  instars
\item
  Model seems to be growing adolescents too fast, but log\_mu\_year
  shrinks them back down \ldots{} an effort to\ldots.?
\end{itemize}

\hypertarget{data}{%
\subsection{Data}\label{data}}

Source data for the model came from the annual snow crab survey from
2006 to 2020.

Average length-frequencies were standardized by trawl swept area were
calculated by survey year, sex and morphometric maturity.

For the purposes of evaluating and developing the snow crab model, the
time series was limited to the period from 2006 to 2020, owing to the
spatial homogeneity of the sampling design during this period. The
survey was marginally expanded in 2011 and this will need to be
considered when interpreting the model results.

The time series will be extended in future version of the model into
past surveys where changing survey area and heterogeneous spatial
distributions may lead to some degree of scaling issues, which will
hopefully be corrected by the model.

This treatment of densities as frequencies allows for approximate
statistical

\hypertarget{model-specification}{%
\subsection{Model specification}\label{model-specification}}

\hypertarget{description}{%
\subsubsection{Description:}\label{description}}

In the literature, benthic stages of snow crab instars are numbered
using roman numerals, with I representing the first stage after the
megalopses larvae have settled on the bottom and moulted.

For both sexes, instars I to VIII are considered sexually immature and
characterized by high relative growth rates. Adolescence begins with the
onset of gonadal development at instar VIII, which is characterized by
lower relative growth rates.

Sexual maturity, equated here as the terminal moult accompanied by
characteristic morphometric changes, is attained at instars IX or
larger. The vast majority of female snow crabs reach sexual maturity at
instars IX and X.

Females growing to instar XI and larger were considered as being too
rare an occurrence to be considered in the analysis. Mature male snow
crab moult to maturity over a much wider size range, from instars IX to
XIII. Instar XIV males were considered as relatively rare and not
considered in the model. It follows that instar X in females and instar
XII in males were the largest adolescent instars. Growth in the model is
a combination of two separate processes: one which specifies the
probability of moulting from one instar to the next, and the other which
specifies the predicted increase in size and its variation when
moulting.

Two moulting processes were considered. Sexual maturation was modelled
as the proportions of crab that undergo the terminal moult to maturity
by instar and year. Although the probability of moulting to maturity for
the largest adolescent instars, i.e.~instar IX in females and XII in
males, was considered to be 1, instars VIII in females and instars
VIII-XI in males show variable proportions from year-to-year.
Skip-moulting was only considered for adolescent males, was similarly
modelled by instar and year. Moulting was considered to occur annually
all instars.

Although the odd instar I, II and III do appear in survey catches, only
instar IV crab are present in sufficient amounts to be analyzed by the
model. For practical reasons, annual recruitment to the population was
defined as the abundance of instar IV.

\hypertarget{formal-definition}{%
\subsubsection{Formal definition:}\label{formal-definition}}

Instar carapace width sizes were assumed to follow a Gaussian
distribution curve with structured mean and error. Means for successive
instars instar were defined iteratively as follows, allowing for known
differences in growth between immature and adolescent crab: \[
   \mu_{k+1} = \alpha_{imm}\mu_{k} + \beta_{imm}, \text{ for }  k \le 8  \\
   \mu_{k+1} = \alpha_{ado}\mu_{k} + \beta_{ado}, \text{ for }  k > 8  \\
\] where \(\mu_{k}\) is the mean for the \(k^{th}\) instar,
\(\alpha_{imm}\) and \(\beta_{ado}\) are Hiatt slope and intercept
parameters, respectively for immature crab, with \(\alpha_{ado}\) and
\(\beta_{ado}\) being the corresponding adolescent phase parameters.

Instar standard errors are similarly defined, but allow for additional
error inflation in the form of two positive parameters \(\gamma_{imm}\)
and \(\gamma_{ado}\):

\[
   \sigma_{k+1} = (\alpha_{imm} + \gamma_{imm}) \sigma_{k}, \text{ for } k \le 8  \text{ and }  \gamma_{imm} > 0 \\
   \sigma_{k+1} = (\alpha_{ado} + \gamma_{ado}) \sigma_{k}, \text{ for } k > 8  \text{ and }  \gamma_{ado} > 0 \\   
\] Growth for mature crab was modified slightly by including an additive
term \(\Delta_{mat}\):

\[
   \mu_{k}^{mat} = \mu_{k}^{ado} + \Delta_{mat}
\] Based on biological considerations, we expect that
\(\Delta_{mat} < 0\).

Inferences on growth-at-moult can often be obtained from analysis of
these modes and the approach has traditionally been to treat the data as
arising from finite mixture model with probability density of the form:

\[
   \sum_{k=1}^{K} \pi_k \phi \left(x | \mu_k , \sigma_k^2  \right)
\] where \(k\) indexes the instars, \(x\) represents crab size,
\(\pi_k\) are the proportions of each instar in the sample, \(\mu_k\)
are their mean sizes and \(\sigma_k^2\) are their variances.

However, inference for larger instars is generally more uncertain owing
to increasing variability in growth during adolescence, which resulting
in size overlap between successive instars at these stages.

\hypertarget{model-assumptions}{%
\subsubsection{Model Assumptions}\label{model-assumptions}}

\begin{itemize}
\item
  There is a largest immature instar, for which all individuals either
  skip-moult or moult to maturity the following year.
\item
  Skip-moulters moult to maturity the following year.
\item
  Skip-moulters only exist from instar IX and onward.
\item
  Females have negligible amounts of skip-moulting.
\item
  Matures exist only from instar IX onward.
\item
\end{itemize}

\hypertarget{population-dynamics-equations}{%
\subsubsection{Population dynamics
equations}\label{population-dynamics-equations}}

Stage-based processes affecting the population dynamics are the
probability of moulting to maturity from one instar to the next, the
probability of an instar skipping a moult (i.e.~not growing and
remaining mature) and annual mortality for immatures and matures.

The selectivity function is length-based, being a sigmoid-type function.

With \(y\) indxing the survey year and \(k\) indexing the instar, the
population dynamics equations are as follows:

\[
\begin{aligned}
   n_{k,y}^{imm}  &= (1-p_{k-1,y-1}^{mat}) \times (1-p_{k-1}^{skip}) \times (1-M^{imm}) \times n_{k-1,y-1}^{imm} \\
   n_{k,y}^{skip} &= (1-p_{k-1,y-1}^{mat}) \times p_{k-1}^{skp} \times (1-M^{imm}) \times n_{k,y-1}^{imm} \\
   n_{k,y}^{rec}  &= (1-M^{mat}) \times \left[(1-p_{k-1}^{skp}) \times p_{k-1,y-1}^{mat} \times n_{k-1,y-1}^{imm} + n_{k-1,y-1}^{skip} \right] \\  
   n_{k,y}^{res}  &= (1-M^{mat}) \times \left[n_{k,y-1}^{rec} + n_{k,y-1}^{res} \right] \\ 
   n_{k,y}^{mat}  &= n_{k,y}^{rec} + n_{k,y}^{res}
\end{aligned}
\] with the superscripts \(imm\) representing regular immatures,
\(skip\) representing immatures which have skipped the previous moult,
\(rec\) representing new mature recruits, \(res\) representing residuals
matures and \(mat\) representing all matures, i.e.~the sum of recruits
and residuals.

\begin{longtable}[]{@{}cl@{}}
\toprule
\begin{minipage}[b]{0.20\columnwidth}\centering
Variable\strut
\end{minipage} & \begin{minipage}[b]{0.74\columnwidth}\raggedright
Description\strut
\end{minipage}\tabularnewline
\midrule
\endhead
\begin{minipage}[t]{0.20\columnwidth}\centering
\(n_{k,y}^{imm}\)\strut
\end{minipage} & \begin{minipage}[t]{0.74\columnwidth}\raggedright
Population number of immature crab.\strut
\end{minipage}\tabularnewline
\begin{minipage}[t]{0.20\columnwidth}\centering
\(n_{k,y}^{skip}\)\strut
\end{minipage} & \begin{minipage}[t]{0.74\columnwidth}\raggedright
Population number of immature crab which skipped the previous
moult.\strut
\end{minipage}\tabularnewline
\begin{minipage}[t]{0.20\columnwidth}\centering
\(n_{k,y}^{rec}\)\strut
\end{minipage} & \begin{minipage}[t]{0.74\columnwidth}\raggedright
Population number of new mature recruits.\strut
\end{minipage}\tabularnewline
\begin{minipage}[t]{0.20\columnwidth}\centering
\(n_{k,y}^{res}\)\strut
\end{minipage} & \begin{minipage}[t]{0.74\columnwidth}\raggedright
Population number of old mature residuals (i.e.~non-recruits).\strut
\end{minipage}\tabularnewline
\begin{minipage}[t]{0.20\columnwidth}\centering
\(n_{k,y}^{mat}\)\strut
\end{minipage} & \begin{minipage}[t]{0.74\columnwidth}\raggedright
Population number of total mature crab.\strut
\end{minipage}\tabularnewline
\begin{minipage}[t]{0.20\columnwidth}\centering
\(M^{imm}\)\strut
\end{minipage} & \begin{minipage}[t]{0.74\columnwidth}\raggedright
Annual proportion of immature crab which die off.\strut
\end{minipage}\tabularnewline
\begin{minipage}[t]{0.20\columnwidth}\centering
\(M^{mat}\)\strut
\end{minipage} & \begin{minipage}[t]{0.74\columnwidth}\raggedright
Annual proportion of mature crab which die off.\strut
\end{minipage}\tabularnewline
\begin{minipage}[t]{0.20\columnwidth}\centering
\(p_{k,y}^{skip}\)\strut
\end{minipage} & \begin{minipage}[t]{0.74\columnwidth}\raggedright
Annual proportion of immature crab which skip a moult.\strut
\end{minipage}\tabularnewline
\begin{minipage}[t]{0.20\columnwidth}\centering
\(p_{k,y}^{mat}\)\strut
\end{minipage} & \begin{minipage}[t]{0.74\columnwidth}\raggedright
Annual proportion of immature crab which moult to maturity.\strut
\end{minipage}\tabularnewline
\bottomrule
\end{longtable}

\newpage

\hypertarget{figures}{%
\subsection{Figures}\label{figures}}

\begin{figure}
\centering
\includegraphics{female model and length-frequencies 2010-2020.pdf}
\caption{Fitted population model abundances for immature (red) and
mature (blue) female snow crab. Jagged curves are observed
length-frequencies.}
\end{figure}

\begin{figure}
\centering
\includegraphics{female trawl selectivity.png}
\caption{Fitted selectivity curve for female snow crab}
\end{figure}

\begin{figure}
\centering
\includegraphics{female moultring probability VIII to IX.png}
\caption{Annual probability of immature instar VIII moulting to mature
IX for female snow crab}
\end{figure}

\hypertarget{selectivity-plot}{%
\section{Selectivity plot:}\label{selectivity-plot}}

Size-selectivity curve showing the estimated proportion of female snow
crab caught by the survey trawl from 2006 to 2020, based on the
population model.

\hypertarget{moulting-probability}{%
\section{Moulting probability:}\label{moulting-probability}}

Probability of the terminal moult to maturity by year for instar VIII in
female snow crab, as estimated from the population model.

\hypertarget{population-abundance-plot}{%
\section{Population abundance plot:}\label{population-abundance-plot}}

Eastimated abundance of immature and mature female snow crab by instar
by year from the population model.

\hypertarget{growth-models}{%
\section{Growth models:}\label{growth-models}}

\hypertarget{year-effect}{%
\section{Year effect}\label{year-effect}}

Estimated scale of year effects for female snow crab from the population
model. Effects are scaled relative to 2020.

\hypertarget{female-model-and-length-frequencies-2010-2020.pdf}{%
\section{female model and length-frequencies
2010-2020.pdf}\label{female-model-and-length-frequencies-2010-2020.pdf}}

Estimated survey size-structured abundance for immature and adolescent
(red lines) and new mature recruits (green lines) and total mature
female snow crab (blue lines). Observed size-frequencies are shown as
jagged lines following the same colour scheme.

\end{document}
